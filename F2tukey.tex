	\emph{John Wilder Tukey} (June 16, 1915 – July 26, 2000) was an American mathematician.	Born in New Bedford, Massachusetts, he earned a B.A. in 1936 and M.Sc. in 1937, in chemistry, from Brown University, before moving to Princeton University where he received a Ph.D. in mathematics. 
	
	During World War II, he worked at the Fire Control Research Office and collaborated with Samuel Wilks and William Cochran. After the war, he returned to Princeton, where he divided his time between the university and AT\&T Bell Laboratories. \emph{He became a full professor at 35 and was founding chairman of the Princeton statistics department in 1965}. 
	
	He was awarded the \emph{National Medal of Science by President Nixon in 1973}, and the IEEE Medal of Honor in 1982 \enquote{For his contributions to the spectral analysis of random processes and the fast Fourier transform (FFT) algorithm}.

	He is known for developing the FFT algorithm, the box plot, the Tukey range test, the Tukey $\lambda$ distribution, the Tukey test of additivity, and the Teichmüller–Tukey lemma. 
	
	He also made many contributions and articulated the important distinction between exploratory data analysis and confirmatory data analysis. In particular, he believed that much statistical methodology placed too great an emphasis on the latter. A. D. Gordon offered the following summary of Tukey's principles for statistics: 
	\begin{itemize}
		\setlength{\itemsep}{0pt}
		\setlength{\parskip}{0pt} 		
		\item The usefulness and limitation of mathematical statistics, 
		\item \emph{the importance of having methods of statistical analysis that are robust to violations of the assumptions underlying their use},  
		\item the need to amass experience of the behavior of specific methods of analysis in order to provide guidance on their use,  
		\item the importance of \emph{allowing the possibility of data's influencing the choice of method by which they are analyzed}, 
		\item the need for statisticians to reject the role of 'guardian of proven truth', and to resist attempts to provide once-for-all solutions and tidy overunifications, 
		\item the iterative nature of data analysis, and 
		\item the importance of the increasing power, availability and cheapness of computers,
		\item with John von Neumann, he introduced the word \enquote{bit} short for \enquote{binary digit}. 
		\item Tukey's 1958 paper \enquote{The Teaching of Concrete Mathematics} contained the earliest known usage of the term \enquote{software}.
	\end{itemize}
	\credits{Source: \url{https://en.wikipedia.org/wiki/John_Tukey}}