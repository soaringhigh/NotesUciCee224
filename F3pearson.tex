	\emph{Karl Pearson} (March 1857 – 27 April 1936) was an English mathematician and biostatistician. He established the discipline of mathematical statistics, and contributed significantly to biometrics, meteorology. Pearson was a protégé and biographer of Sir Francis Galton, who coined the term \enquote{regression}.
	
	After a private education at University College School, he went to King's College, Cambridge in 1876 to study mathematics. He then traveled to Germany to study physics and metaphysics in Heidelberg. He attended lectures on Darwinism, but also Roman law, medieval and 16th century German literature, and Socialism in Berlin.
	
	After returning to London, he studied law until 1881 but never practiced. He then returned to mathematics, first at King's College, London in 1881 and then at University College, London in 1883. After his appointment to the professorship of Geometry at Gresham College, he met Walter Frank Raphael Weldon, a zoologist with whom he developed a fruitful collaboration. Weldon introduced him to Darwin's cousin Francis Galton. Pearson became Galton's protégé, and after Galton's death in 1911, he wrote his biography. He formed the Department of Applied Statistics, into which he incorporated the Biometric and Galton laboratories. In 1890, he married Maria Sharpe, and they had three children. 
	
	Unfortunately, Pearson was racist, anti‐Semitic, and a proponent of eugenics, i.e., a social philosophy advocating the improvement of human genetic traits through the promotion of higher rates of reproduction for people with desired traits, and reduced rates of reproduction and sterilization of people with less‐desired or undesired traits. 

	Pearson's work embraced wide applications and the development of mathematical statistics, with contributions to \emph{biology, epidemiology, anthropometry, medicine, psychology and social history}. In 1901, with Weldon and Galton, he founded the journal Biometrika that focuses on statistical theory. Pearson's thinking underpins many of the 'classical' methods which are in common use today. Some contributions are:
	\begin{itemize}
		\setlength{\itemsep}{0pt}
		\setlength{\parskip}{0pt} 		
		\item Correlation coefficient. He studied its relationship with  linear regression,
		\item method of moments: Pearson introduced the concept borrowed from physics,
		\item Pearson's system of continuous univariate probability distributions that came to form the basis of the now conventional continuous probability distributions,
		\item foundations of hypothesis testing and of the statistical decision theory,
		\item use of P-values and Pearson's chi-squared test,
		\item Principal component analysis. 
	\end{itemize}
	\credits{Source: \url{https://en.wikipedia.org/wiki/Karl_Pearson}}