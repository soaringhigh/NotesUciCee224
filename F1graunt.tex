\emph{John Graunt} (April 24, 1620 – April 18, 1674) was \emph{one of the first demographers}, though by profession he was a haberdasher (i.e., he sold small articles for sewing, such as buttons, ribbons, zips).
	
	He was born in London, the eldest of seven or eight children of Henry and Mary Graunt. His father was a draper who had moved to London from Hampshire. In February 1641, John Graunt married Mary Scott, with whom he had one son and three daughters.
	
	He worked in his father's shop until his father died in 1662, and became influential in the City. He served in various ward offices, becoming a common councilman about 1669–71, warden of the Drapers' Company in 1671 and a major in the trained band.
	
	His house was destroyed in the Great Fire of London and he encountered other financial problems leading eventually to bankruptcy. His daughter became a nun in a Belgian convent and Graunt decided to convert to Catholicism at a time when Catholics and Protestants were struggling for control of England and Europe, leading to prosecutions for recusancy. He died of jaundice and liver disease at the age of 53.
	
	With William Petty, \emph{he developed early human statistical and census methods that later provided a framework for modern demography}. He is credited with producing the first life table, giving probabilities of survival to each age. In addition, he is considered one of the first experts in epidemiology, since his famous book was concerned mostly with public health statistics.
	
	His book \enquote{Natural and Political Observations Made upon the Bills of Mortality} (1663) used analysis of the mortality rolls in early modern London as Charles II and other officials attempted to create a system to warn of the onset and spread of bubonic plague in the city. Though his system was not fully created, Graunt's work resulted in the first statistically based estimation of the population of London.
	
	He presented his work to the Royal Society and was subsequently elected a fellow in 1662 with the endorsement of the King. 
	\credits{Source: \url{https://en.wikipedia.org/wiki/John_Graunt}}