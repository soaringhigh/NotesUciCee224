\emph{Johann Carl Friedrich Gauß} (30 April 1777 – 23 February 1855) in Brunswick, in the Duchy of Brunswick-Wolfenbüttel (now part of Lower Saxony, Germany, as the son of poor working-class parents. \emph{Gauß was a child prodigy}. A contested story relates that, when he was eight, he figured out how to add up all the numbers from 1 to 100. 

Gauß' intellectual abilities attracted the attention of the Duke of Brunswick, who sent him to the Collegium Carolinum (1792 to 1795, and to the University of Göttingen (1795 to 1798). While there, Gauß independently rediscovered several important theorems. In 1796, he became the first to prove the quadratic reciprocity law, which allows mathematicians to \emph{determine the solvability of any quadratic equation}. He also conjectured the prime number theorem, which gives a good understanding of how prime numbers are distributed among integers. In his 1799 doctorate, Gauß proved \emph{that every nonconstant single-variable polynomial with complex coefficients has at least one complex root}.  Gauß also made important contributions to number theory in his 1801 book Disquisitiones Arithmeticae.

In 1831 Gauß developed a fruitful collaboration with physicist Wilhelm Weber, leading to new knowledge in magnetism and the \emph{discovery of Kirchhoff's electric circuit laws}. They constructed the first electromechanical telegraph in 1833, which connected the observatory with the institute for physics in Göttingen. In 1840, Gauß published his influential \enquote{Dioptrische Untersuchungen}, in which he gave the first systematic analysis on the formation of images under a paraxial approximation (\emph{Gaußian optics}).

Gauß' personal life was overshadowed by the early death of his first wife, Johanna Osthoff, in 1809, soon followed by the death of one child, which caused him to become depressed. When his second wife died in 1831 after a long illness, one of his daughters took over the household and cared for Gauß for the rest of his life. Gauß had six children, 3 with each wife. 
 
Gauß made major contributions to various areas of mathematics (including geometry and number theory) and physics (including magnetism and optics). The importance of his contributions is often compared to those of Newton. He also made some key contributions to statistics. The most important one is the development of \emph{least squares estimation recursive methods}, which he discusses in a book on planetary orbits. He also proposed some, which are used for time series analysis and were used to help calculate the trajectory of the Apollo spacecraft. He is also \emph{credited with developing the normal distribution} (also called the Gaussian distribution or bell curve), which is extremely useful in probability and statistics.
\credits{Sources: \url{https://en.wikipedia.org/wiki/Carl_Gauss}, \url{http://www.sciencedirect.com/science/article/pii/0315086078900496}}